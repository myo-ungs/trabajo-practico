\documentclass[a4paper,12pt]{article}
\usepackage[utf8]{inputenc}
\usepackage[T1]{fontenc}
\usepackage[spanish]{babel}
\usepackage{amsmath,amssymb,amsthm}
\usepackage{amsfonts}
\usepackage{enumitem}
\usepackage{geometry}
\geometry{margin=2.5cm}
\usepackage{fancyhdr}
\usepackage{graphicx}
\usepackage{listings}
\usepackage{xcolor}
\usepackage{longtable}
\usepackage{hyperref}
\hypersetup{
    colorlinks=true,
    linkcolor=blue,
    filecolor=magenta,      
    urlcolor=cyan,
    pdftitle={Optimización de Picking},
    pdfpagemode=FullScreen,
}

\begin{document}

\begin{titlepage}
    \centering
    \vspace*{0.9in}
    
    % Título del informe
    {\Huge \bfseries Informe de 1er Entrega \par}
    
    \vspace{1in}
    
    % Universidad y materia
    {\LARGE \bfseries Universidad Nacional General Sarmiento \par}
    \vspace{0.3in}
    {\Large \textit{Modelado y Optimización} \par}
    \vspace{1in}
    {\Large \bfseries Profesor: \par}
    \vspace{0.3in}
    {\Large Marcelo Mydlarz \par}

    \vspace{0.6in}
    
    % Integrantes
    {\Large \bfseries Integrantes: \par}
    \vspace{0.3in}
    {\Large Juan Manuel Losada \par}
    \vspace{0.1in}
    {\Large Facundo Ruiz \par}
    \vspace{0.1in}
    {\Large Matías Morales \par}
    \vspace{0.1in}
    {\Large Vanesa Vera \par}

    \vspace{1in}
    
    % Fecha
    % Pie de página opcional
    \vfill
    {\Large 17 de Junio 2025 \par}
    
\end{titlepage}

% Configuración de encabezados y pies de página
\pagestyle{fancy}
\fancyhf{}
\lhead{Modelado y Optimización}
\rhead{\thepage}
\cfoot{}

% Definición de entornos
\newtheorem{theorem}{Teorema}
\newtheorem{lemma}{Lema}
\newtheorem{definition}{Definición}
\newtheorem{problem}{Problema}

% Configuración de listings
\lstdefinestyle{PythonStyle}{
  language=Python,
  basicstyle=\ttfamily\footnotesize,
  keywordstyle=\color{blue}\bfseries,
  commentstyle=\color{green!60!black}\itshape,
  stringstyle=\color{red!80!black},
  numberstyle=\tiny\color{gray},
  showstringspaces=false,
  tabsize=2,
  breaklines=true,
  breakatwhitespace=true,
  numbers=left,
  frame=tb,
  framesep=5pt,
  framexleftmargin=2pt,
  captionpos=b,
  extendedchars=true,
  inputencoding=utf8,
  literate={á}{{\'a}}1 {é}{{\'e}}1 {í}{{\'i}}1 {ó}{{\'o}}1 {ú}{{\'u}}1 {Á}{{\'A}}1 {É}{{\'E}}1 {Í}{{\'I}}1 {Ó}{{\'O}}1 {Ú}{{\'U}}1 {ñ}{{\~n}}1 {Ñ}{{\~N}}1
}
\lstset{style=PythonStyle}


%====================================================================
% CARÁTULA
%====================================================================
\thispagestyle{empty}
\clearpage

\tableofcontents
\clearpage

%====================================================================
\section{Introducción}
%====================================================================



\subsection{Definiciones importantes}

\subsubsection*{Entrada en Calor:}

\begin{itemize}
    \item $I$: conjunto de ítems.
    \item $O$ de bolsitas. Cada bolsita $o \in O$ contiene ítems del conjunto $I$. El ítem $i \in I$ aparece $u_{oi}$ veces en la bolsita $o$.
    \item $A$: conjunto de contenedores. Cada contenedor $a \in A$ contiene ítems del conjunto $I$. El ítem $i \in I$ aparece $u_{ai}$ veces en el contenedor $a$.
    \item $b_{oi}$: beneficio del ítem $i$ en la bolsita $o$.
    \item $b_a$: beneficio asociado al contenedor $a$.
    
\end{itemize}

\subsubsection*{El Desafío:}

\begin{itemize}
    \item $O$: conjunto de órdenes.
    \item $I$: conjunto de elementos.
    \item $A$: conjunto de pasillos.
    \item $LB$: límite inferior de unidades a recolectar.
    \item $UB$: límite superior de unidades a recolectar.
\end{itemize}

\subsection{Parámetros del Sistema}

\subsubsection*{Entrada en Calor:} 

\begin{itemize}
    \item $c_{jt}$: Cantidad del producto $t$ requerida por la orden $j$
    \item $d_{kt}$: Cantidad del producto $t$ disponible en el pasillo $k$
    \item $\beta_j$: Beneficio obtenido al seleccionar la orden $j$
    \item $\alpha_k$: Costo de utilizar el pasillo $k$
    \item $L_w, U_w$: Límites inferior y superior para la ola de trabajo (wave), que definen el rango permitido para la cantidad total de productos procesados
\end{itemize}

\subsubsection*{El Desafío:}

\begin{itemize}
    \item $c_{jt}$: Cantidad del producto $t$ requerida por la orden $j$
    \item $d_{kt}$: Cantidad del producto $t$ disponible en el pasillo $k$

\end{itemize}

\subsection{Formato de Datos de Entrada}

Los datos del problema se proporcionan en archivos de texto estructurados con el siguiente formato:

\subsubsection*{Entrada en Calor:}

\begin{itemize}
    \item \textbf{Primera línea:} contiene tres enteros $o$, $i$, $a$ que indican la cantidad de bolsitas, el número de items diferentes y la cantidad de contenedores, respectivamente.
    \item \textbf{Siguientes $o$ líneas:} cada línea comienza con un entero \(k\), seguido por \(k\) pares de enteros, representando el número del item y la cantidad de veces que aparece en la bolsita. Dentro de este grupo de líneas, y para \(j \in \{1, \ldots, o\}\), la \(j\)-ésima línea contiene datos relacionados con la bolsita indexada por \(j-1\), significando que las bolsitas están numeradas de \(0\) a \(o-1\). Además, se asume que los items están indexados de \(0\) a \(i-1\).
    \item \textbf{Siguientes $J$ líneas:} cada línea comienza con un entero \(l\), seguido por \(l\) pares de enteros, representando el número del item y el número de unidades disponibles en el contenedor. Similar al caso de las bolsitas, dentro de este grupo de líneas, y para \(j \in \{1, \ldots, a\}\), la \(j\)-ésima línea contiene datos relacionados con el contenedor indexado por \(j-1\), significando que los contenedores están numerados de \(0\) a \(a-1\).
\end{itemize}

\subsubsection*{El Desafío:}

\begin{itemize}
    \item \textbf{Primera línea:} contiene tres enteros $o$, $i$, $a$ que indican la cantidad de órdenes, la cantidad de elementos diferentes y la cantidad de pasillos, respectivamente.

    \item \textbf{Siguientes $o$ líneas:} cada línea representa una orden. 
    \begin{itemize}
        \item Cada línea comienza con un entero $k$, que indica la cantidad de elementos distintos solicitados en la orden.
        \item Luego siguen $k$ pares de números $(e_j, q_j)$ donde:
        \begin{itemize}
            \item $e_j$ es el índice del elemento solicitado.
            \item $q_j$ es la cantidad requerida del elemento $e_j$.
        \end{itemize}
    \end{itemize}

    \item \textbf{Siguientes $a$ líneas:} cada línea representa un pasillo.
    \begin{itemize}
        \item Cada línea comienza con un entero $l$, que indica la cantidad de tipos de elementos disponibles en ese pasillo.
        \item Luego siguen $l$ pares de números $(e_k, s_k)$ donde:
        \begin{itemize}
            \item $e_k$ es el índice del elemento disponible.
            \item $s_k$ es la cantidad disponible del elemento $e_k$ en ese pasillo.
        \end{itemize}
    \end{itemize}

    \item \textbf{Última línea:} contiene dos enteros $LB$ y $UB$, que representan el límite inferior y superior de unidades a recolectar, respectivamente.
\end{itemize}

\subsection{Descripción de problemas:}

\subsubsection*{Entrada en Calor:}

\begin{enumerate}
    \item \textbf{Problema 1: Encontrar $a$ }
    
     Este problema modela una situación en la que se desea elegir un único contenedor que debe abastecer un conjunto de bolsitas. El desafío está en elegir la mejor combinación posible de bolsitas que pueden ser cubiertas por ese contenedor, de forma tal que el beneficio total sea máximo.
    
    \item \textbf{Problema 2: Encontrar $A'$ }
    
    Este problema generaliza el anterior: ahora se permite seleccionar más de un contenedor, lo cual da más flexibilidad, pero también puede hacer más complejo encontrar la mejor solución. El objetivo sigue siendo cubrir completamente las demandas de las bolsitas con la oferta de los contenedores, maximizando el beneficio total del sistema.
    
\end{enumerate}

\subsubsection*{El Desafío:}

\begin{enumerate}
    \item \textbf{Parte 1: Desafío(\(|A'|\))}
    
    En esta etapa del trabajo se aborda el problema Desafío(|A′|), donde se fija la cantidad de pasillos que pueden visitarse, sin especificar cuáles. El objetivo es modelar e implementar una formulación que permita seleccionar de forma óptima los pasillos a utilizar, respetando dicha cantidad máxima. Además, se requiere desarrollar una función que verifique si una solución es factible, es decir, si cumple con las restricciones del problema, como la cobertura adecuada de órdenes y la capacidad limitada de los pasillos.
    
    
    \item \textbf{Parte 2: Desafío(\(A'\))}
    
    En esta etapa del trabajo se aborda el problema Desafío(|A′|), donde se fija la cantidad de pasillos que pueden visitarse, sin especificar cuáles. El objetivo es modelar e implementar una formulación que permita seleccionar de forma óptima los pasillos a utilizar, respetando dicha cantidad máxima. Además, se requiere desarrollar una función que verifique si una solución es factible, es decir, si cumple con las restricciones del problema, como la cobertura adecuada de órdenes y la capacidad limitada de los pasillos.

    
    \item \textbf{Parte 3: Generación de columnas para Desafío(\(|A'|\)) }
    
    En esta tercera parte se aborda la resolución del problema mediante un enfoque de \textbf{generación de columnas}, lo cual permite enfrentar instancias complejas de gran tamaño. La idea consiste en representar posibles decisiones como \textit{patrones}, cada uno formado por un par $(O, a)$, donde $O$ es un subconjunto de órdenes que pueden ser satisfechas exclusivamente utilizando el pasillo $a$. Estos patrones se codifican como vectores binarios de tamaño $|O| + |\mathcal{A}|$, indicando qué órdenes y qué pasillo participan en ese patrón. El modelo no considera todas las combinaciones posibles desde el inicio, sino que parte de un conjunto reducido y va incorporando nuevas columnas (patrones) sólo cuando tienen potencial de mejorar la solución. Este enfoque requiere desarrollar funciones específicas para construir el modelo inicial (\texttt{ConstruirModelo(...)}), y para agregar nuevos patrones (\texttt{AgregarColumna(...)}), garantizando que las combinaciones elegidas cumplan con todas las restricciones del problema.


    \item \textbf{Parte 4: Resolución completa del problema Desafío}

    En esta etapa se aborda la resolución integral del problema \textit{Desafío}, integrando las estrategias desarrolladas en las partes anteriores. Para ello, se implementa el método \texttt{Opt\_ExplorarCantidadPasillos(umbral)}, que ejecuta de forma iterativa el modelo propuesto en la sección~\ref{sec:variante1} (función \texttt{Opt\_cantidadPasillosFija}), donde se considera una cantidad fija \(k\) de pasillos a seleccionar.
    
    Los valores de \(k\) se determinan a partir de la función \texttt{Rankear}, que estima el potencial de cada valor en función de la capacidad acumulada de los pasillos. Por cada \(k\), se ejecuta el modelo de la sección~\ref{sec:variante1} dentro del tiempo asignado, y se conserva la mejor solución encontrada.
    
    Una vez finalizada la etapa exploratoria, se fijan los pasillos de la mejor solución obtenida y se aplica el modelo de la sección~\ref{sec:variante2} (función \texttt{Opt\_PasillosFijos}), que refina la selección de órdenes en base a la disponibilidad real de los pasillos previamente seleccionados.
    
    Este enfoque permite encontrar una solución final más eficiente, combinando exploración amplia (con múltiples valores de \(k\)) y una etapa de mejora local. Se garantiza que el tiempo total de ejecución del método \texttt{Opt\_ExplorarCantidadPasillos(umbral)} no exceda el umbral de tiempo especificado.


    

    \item \textbf{Parte 5: }
    
    
\end{enumerate}

\clearpage

%====================================================================
\section{Entrada en Calor}
%====================================================================

Esta sección presenta el desarrollo matemático detallado de los modelos para los problemas relacionados con la entrada en calor.

\subsection{Problema 1:}
\label{sec:probl1}

\subsubsection{Formulación Matemática}

\textbf{Variables:}
\begin{itemize}
    \item $x_a \in \{0,1\}$: 1 si el contenedor $a \in A$ es seleccionado, 0 en caso contrario
    \item $y_o \in \{0,1\}$: 1 si la bolsita $o \in O$ es seleccionada, 0 en caso contrario
\end{itemize}

\textbf{Parámetros:}
\begin{itemize}
    \item $B_a$: beneficio del contenedor $a$
    \item $B_{oi}$: beneficio de la bolsita $o$ para el ítem $i$
    \item $U_{oi}$: cantidad de ítems del tipo $i$ en la bolsita $o$
    \item $U_{ai}$: cantidad de ítems del tipo $i$ en el contenedor $a$
\end{itemize}

\textbf{Función Objetivo:}
\begin{equation}
\max \quad \sum_{a \in A} B_a \, x_a + \sum_{o \in O} \sum_{i} B_{oi} \, y_o
\end{equation}

\textbf{Restricciones:}
\begin{align}
& \sum_{a \in A} x_a = 1 \\[6pt]
& \sum_{a \in A} U_{ai} \, x_a \geq \sum_{o \in O} U_{oi} \, y_o, \quad \forall i \\
& x_a \in \{0,1\}, \quad \forall a \in A \\
& y_o \in \{0,1\}, \quad \forall o \in O
\end{align}

\clearpage

\subsection{Problema 2:}
\label{sec:probl2}

\subsubsection{Formulación Matemática}
\textbf{Variables:}
\begin{itemize}
    \item $x_a \in \{0,1\}$: 1 si el contenedor $a \in A$ es seleccionado, 0 en caso contrario
    \item $y_o \in \{0,1\}$: 1 si la bolsita $o \in O$ es seleccionada, 0 en caso contrario
\end{itemize}

\textbf{Parámetros:}
\begin{itemize}
    \item $B_a$: beneficio del contenedor $a$
    \item $B_{oi}$: beneficio de la bolsita $o$ para el ítem $i$
    \item $U_{oi}$: cantidad de ítems del tipo $i$ en la bolsita $o$
    \item $U_{ai}$: cantidad de ítems del tipo $i$ en el contenedor $a$
\end{itemize}

\textbf{Función Objetivo:}
\begin{equation}
\max \quad \sum_{a \in A} B_a x_a + \sum_{o \in O} \sum_{i} B_{oi} y_o
\end{equation}

\textbf{Restricciones:}
\begin{equation}
\sum_{a \in A} x_a = 1
\end{equation}

\begin{equation}
\sum_{a \in A} U_{ai} x_a \geq \sum_{o \in O} U_{oi} y_o, \quad \forall i
\end{equation}

\begin{equation}
x_a \in \{0,1\}, \quad \forall a \in A
\end{equation}

\begin{equation}
y_o \in \{0,1\}, \quad \forall o \in O
\end{equation}

\clearpage

%====================================================================
\section{El Desafió}
%====================================================================

Esta sección presenta el desarrollo matemático detallado de los modelos y sus implementación en Python para la los problemas del desafió.

\subsection{Primera parte:}
\label{sec:variante1}

\subsubsection{Formulación Matemática}

\textbf{Conjuntos e Índices:}
\begin{itemize}
    \item $O = \{0, 1, \ldots, o{-}1\}$: Conjunto de órdenes
    \item $I = \{0, 1, \ldots, i{-}1\}$: Conjunto de elementos o productos
    \item $A$: Conjunto de pasillos disponibles
\end{itemize}

\textbf{Parámetros:}
\begin{itemize}
    \item $W_{io}$: Cantidad del elemento $i$ requerido por la orden $o$
    \item $S_{ia}$: Cantidad del elemento $i$ disponible en el pasillo $a$
    \item $LB$: Límite inferior de unidades a recolectar
    \item $UB$: Límite superior de unidades a recolectar
    \item $P$: Cantidad de pasillos a seleccionar
\end{itemize}

\textbf{Variables de Decisión:}
\begin{itemize}
    \item $x_o \in \{0,1\}$: Toma valor 1 si la orden $o$ es seleccionada, 0 en caso contrario
    \item $y_a \in \{0,1\}$: Toma valor 1 si el pasillo $a$ es seleccionado, 0 en caso contrario
\end{itemize}

\textbf{Función Objetivo:}
\begin{equation}
\max \quad \frac{1}{P} \sum_{o \in O} \sum_{i \in I} W_{io} \cdot x_o
\end{equation}

\textbf{Restricciones:}
\begin{align}
\sum_{o \in O} \sum_{i \in I} W_{io} \cdot x_o &\geq LB \label{eq:limite_inferior} \\[6pt]
\sum_{o \in O} \sum_{i \in I} W_{io} \cdot x_o &\leq UB \label{eq:limite_superior} \\[6pt]
\sum_{o \in O} W_{io} \cdot x_o &\leq \sum_{a \in A} S_{ia} \cdot y_a, \quad \forall i \in I \label{eq:disponibilidad} \\[6pt]
\sum_{a \in A} y_a &= P \label{eq:pasillos_fijos}
\end{align}

\subsubsection{Implementación en Python}

La implementación se encuentra en \texttt{parte1.py} y utiliza PuLP para la optimización:

\begin{lstlisting}[language=Python, caption=Implementación clave de la parte 1]

\end{lstlisting}

\clearpage

\subsection{Segunda parte:}
\label{sec:variante2}

\subsubsection{Formulación Matemática}

\textbf{Conjuntos:}
\begin{itemize}
    \item $O$: conjunto de órdenes
    \item $I$: conjunto de elementos
    \item $A_0 \subseteq A$: conjunto de pasillos preseleccionados
    \item $A'$: pasillos a visitar
\end{itemize}

\textbf{Parámetros:}
\begin{itemize}
    \item $W_{io}$: cantidad del elemento $i$ requerido por la orden $o$
    \item $S_{ia}$: cantidad del elemento $i$ disponible en el pasillo $a$
    \item $LB$: límite inferior de unidades a recolectar
    \item $UB$: límite superior de unidades a recolectar
\end{itemize}

\textbf{Variables de Decisión:}
\begin{itemize}
    \item $x_o \in \{0,1\}$: Toma valor 1 si la orden $o$ es seleccionada, 0 en caso contrario
\end{itemize}

\textbf{Función Objetivo:}
\begin{equation}
\operatorname{max} \quad \sum_{o \in O} \sum_{i \in I} W_{io} \cdot x_o
\end{equation}


\textbf{Restricciones:}
\begin{align}
\sum_{o \in O} \sum_{i \in I} W_{io} \cdot x_o &\geq LB \\
\sum_{o \in O} \sum_{i \in I} W_{io} \cdot x_o &\leq UB \\
\sum_{o \in O} W_{io} \cdot x_o &\leq \sum_{a \in A'} S_{ia}, \quad \forall i \in I
\end{align}

\subsubsection{Implementación en Python}

La implementación en \texttt{parte2.py} incluye validaciones para verificar que las órdenes fijas cumplan con los límites de ola de trabajo:

\begin{lstlisting}[language=Python, caption=Implementación clave de la segunda parte]

\end{lstlisting}

\clearpage

\subsection{Tercera parte:}
\label{sec:variante3}

\subsubsection{Formulación Matemática}

\textbf{Conjuntos:}
\begin{itemize}
    \item $O = \{0, \ldots, n-1\}$: conjunto de órdenes
    \item $P = \{0, \ldots, m-1\}$: conjunto de pasillos
    \item $C$: conjunto de columnas válidas (índices $i$)
\end{itemize}

\textbf{Parámetros:}
\begin{itemize}
    \item $c_{io} \in \{0,1\}$: indica si la columna $i$ cubre la orden $o$
    \item $a_i \in P$: pasillo asociado a la columna $i$
    \item $M$: número máximo de pasillos a seleccionar
\end{itemize}

\textbf{Variables de Decisión:}
\begin{itemize}
    \item $z_i \in \{0,1\}$: 1 si se selecciona la columna $i$, 0 en caso contrario
    \item $y_a \in \{0,1\}$: 1 si se selecciona el pasillo $a$, 0 en caso contrario
\end{itemize}

\textbf{Función Objetivo:}
\begin{equation}
\max \quad \sum_{i \in C} \left(\sum_{o \in O} c_{io}\right) z_i
\end{equation}

\textbf{Restricciones:}
\begin{align}
&\sum_{i \in C} c_{io} \, z_i \leq 1, \quad \forall o \in O \\
&z_i \leq y_{a_i}, \quad \forall i \in C \\
&\sum_{a \in P} y_a = M
\end{align}


\subsubsection{Implementación en Python}

El archivo \texttt{parte3.py} implementa el algoritmo completo:

\begin{lstlisting}[language=Python, caption=Algoritmo principal de generación de columnas]

\end{lstlisting}

\clearpage

\subsection{Cuarta parte:}
\label{sec:variante4}


\subsubsection{Formulación Matemática}

En esta parte se combinan dos modelos de optimización previamente definidos:

\begin{itemize}
    \item El modelo de la \textbf{Primera parte} (sección~\ref{sec:variante1}) es utilizado por la función \\ \texttt{Opt\_cantidadPasillosFija(k, umbral)}. 
    \item El modelo de la \textbf{Segunda parte} (sección~\ref{sec:variante2}) es utilizado por la función \\ \texttt{Opt\_PasillosFijos(umbral)}.
\end{itemize}

Ambos modelos son de programación lineal entera binaria y comparten la función objetivo de maximizar la cobertura de órdenes. Se reutilizan los mismos conjuntos, parámetros y variables definidos en las secciones anteriores.

\subsubsection{Algoritmo General}

El procedimiento llevado a cabo por 
\texttt{Opt\_ExplorarCantidadPasillos(umbral)} sigue estos pasos:

\begin{enumerate}
    \item Se ejecuta la función \texttt{Rankear} para priorizar valores de \(k\) según el potencial de cobertura de cada cantidad de pasillos.
    \item Para cada \(k\) priorizado, se ejecuta \texttt{Opt\_cantidadPasillosFija(k, tiempo)} (modelo de la sección~\ref{sec:variante1}), hasta alcanzar el límite de tiempo.
    \item Se selecciona la mejor solución hallada y se extraen los pasillos utilizados.
    \item Con estos pasillos fijados, se ejecuta \texttt{Opt\_PasillosFijos(tiempo restante)} (modelo de la sección~\ref{sec:variante2}) para refinar la selección de órdenes.
\end{enumerate}
\subsubsection{Implementación en Python}

El archivo \texttt{parte4.py} contiene la implementación completa del procedimiento descrito. A continuación se muestra un esquema simplificado:

\begin{lstlisting}[language=Python, caption=Algoritmo general de exploración y refinamiento]

\end{lstlisting}

\clearpage

\subsection{Quinta parte:}
\label{sec:variante5}


\subsubsection{Formulación Matemática}

\subsubsection{Algoritmo General}

\clearpage

\section{Herramientas y Tecnologías Utilizadas}

\begin{itemize}
    \item \textbf{Python}: Lenguaje de programación principal
    \item \textbf{PuLP}: Librería de programación lineal que proporciona una interfaz de alto nivel para solvers de optimización
    \item \textbf{SCIP}: Modificar

\end{itemize}

%====================================================================
\section{Dificultades y aprendizajes}
%====================================================================

Durante el desarrollo de este trabajo se enfrentaron diversas dificultades que proporcionaron valiosos aprendizajes sobre optimización combinatoria y su implementación práctica.

\begin{enumerate}
    \item \textbf{Comprensión del contexto del problema}
    
    La transición conceptual desde la descripción abstracta del problema hacia su interpretación en el contexto real de optimización de almacenes requirió un análisis cuidadoso de la semántica de variables y restricciones.
    
    \emph{Aprendizaje:} La importancia de contextualizar problemas de optimización en dominios específicos para una mejor comprensión y comunicación de resultados.


    \item \textbf{Complejidad de la Generación de Columnas}
    
    Modificar
    
    \emph{Aprendizaje:} 

\end{enumerate}

\textbf{Principales aprendizajes obtenidos:}

\begin{itemize}
    \item \textbf{SCIP}: 
    
\end{itemize}

\end{document}
