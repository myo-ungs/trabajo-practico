\documentclass[a4paper,12pt]{article}
\usepackage[utf8]{inputenc}
\usepackage[T1]{fontenc}
\usepackage[spanish]{babel}
\usepackage{amsmath,amssymb,amsthm}
\usepackage{amsfonts}
\usepackage{enumitem}
\usepackage{geometry}
\geometry{margin=2.5cm}
\usepackage{fancyhdr}
\usepackage{graphicx}
\usepackage{listings}
\usepackage{xcolor}
\usepackage{longtable}
\usepackage{hyperref}
\hypersetup{
    colorlinks=true,
    linkcolor=blue,
    filecolor=magenta,      
    urlcolor=cyan,
    pdftitle={Optimización de Picking},
    pdfpagemode=FullScreen,
}

\title{Optimización de Picking en Almacenes:\\[1ex]
       \large Modelado Matemático y Resolución de Variantes del Problema}
\author{Juan Manuel Losada \\\\ Facundo Ruiz \\\\ Matías Morales \\\\ Vanesa Vera}
\date{\today}

% Configuración de encabezados y pies de página
\pagestyle{fancy}
\fancyhf{}
\lhead{Optimización de Picking en Almacenes}
\rhead{\thepage}
\cfoot{}

% Definición de entornos
\newtheorem{theorem}{Teorema}
\newtheorem{lemma}{Lema}
\newtheorem{definition}{Definición}
\newtheorem{problem}{Problema}

% Configuración de listings
\lstdefinestyle{PythonStyle}{
  language=Python,
  basicstyle=\ttfamily\footnotesize,
  keywordstyle=\color{blue}\bfseries,
  commentstyle=\color{green!60!black}\itshape,
  stringstyle=\color{red!80!black},
  numberstyle=\tiny\color{gray},
  showstringspaces=false,
  tabsize=2,
  breaklines=true,
  breakatwhitespace=true,
  numbers=left,
  frame=tb,
  framesep=5pt,
  framexleftmargin=2pt,
  captionpos=b,
  extendedchars=true,
  inputencoding=utf8,
  literate={á}{{\'a}}1 {é}{{\'e}}1 {í}{{\'i}}1 {ó}{{\'o}}1 {ú}{{\'u}}1 {Á}{{\'A}}1 {É}{{\'E}}1 {Í}{{\'I}}1 {Ó}{{\'O}}1 {Ú}{{\'U}}1 {ñ}{{\~n}}1 {Ñ}{{\~N}}1
}
\lstset{style=PythonStyle}

\begin{document}

%====================================================================
% CARÁTULA
%====================================================================
\maketitle
\thispagestyle{empty}
\clearpage

\tableofcontents
\clearpage

%====================================================================
\section{Introducción}
%====================================================================

En el contexto de la optimización combinatoria, uno de los problemas fundamentales es la asignación eficiente de recursos limitados para maximizar beneficios o minimizar costos. Este trabajo se enfoca en un problema específico de optimización de picking en almacenes que involucra la coordinación entre dos tipos de entidades: \emph{órdenes} que contienen productos requeridos y \emph{pasillos} que almacenan productos disponibles.

El problema central consiste en seleccionar subconjuntos de órdenes y pasillos de manera que se maximice el beneficio total, respetando las restricciones de disponibilidad de productos y los límites de ola de trabajo del sistema. Este tipo de problema surge frecuentemente en aplicaciones de logística de almacenes, gestión de inventarios, y operaciones de fulfillment en e-commerce.

\subsection{Descripción del Problema}

Consideramos un sistema de almacén con los siguientes elementos fundamentales:

\begin{itemize}
    \item \textbf{Productos ($I$):} Un conjunto de $i$ tipos diferentes de productos. Cada producto se identifica por un índice único $t \in \{0, 1, \ldots, i-1\}$.
    \item \textbf{Órdenes ($O$):} Un conjunto de $o$ órdenes, donde cada orden $j \in \{0, 1, \ldots, o-1\}$ contiene una cantidad específica $c_{jt}$ de cada tipo de producto $t$. Las órdenes representan demandas de clientes que deben ser satisfechas.
    \item \textbf{Pasillos ($A$):} Un conjunto de $a$ pasillos de almacén, donde cada pasillo $k \in \{0, 1, \ldots, a-1\}$ almacena una cantidad específica $d_{kt}$ de cada tipo de producto $t$. Los pasillos representan los recursos de inventario disponibles.
\end{itemize}

\subsection{Parámetros del Sistema}

El sistema se caracteriza por los siguientes parámetros:

\begin{itemize}
    \item $c_{jt}$: Cantidad del producto $t$ requerida por la orden $j$
    \item $d_{kt}$: Cantidad del producto $t$ disponible en el pasillo $k$
    \item $\beta_j$: Beneficio obtenido al seleccionar la orden $j$
    \item $\alpha_k$: Costo de utilizar el pasillo $k$
    \item $L_w, U_w$: Límites inferior y superior para la ola de trabajo (wave), que definen el rango permitido para la cantidad total de productos procesados
\end{itemize}

\subsection{Formato de Datos de Entrada}

Los datos del problema se proporcionan en archivos de texto estructurados con el siguiente formato:

\begin{itemize}
    \item \textbf{Primera línea:} \texttt{K I J}, donde $K$ es el número de órdenes, $I$ es el número de tipos de productos, y $J$ es el número de pasillos
    \item \textbf{Siguientes $K$ líneas:} Cada línea describe una orden con formato \texttt{orden\_id num\_productos producto\_id:cantidad ...}
    \item \textbf{Siguientes $J$ líneas:} Cada línea describe un pasillo con formato \texttt{pasillo\_id coordenada\_x coordenada\_y num\_productos producto\_id:cantidad ...}
\end{itemize}

\subsection{Variantes del Problema Abordadas}

Este trabajo desarrolla soluciones para tres variantes específicas del problema:

\begin{enumerate}
    \item \textbf{Variante 1: Selección de Órdenes con Pasillos Fijos}
    
    Dado un conjunto predefinido de pasillos $A_{fijo} \subseteq A$, seleccionar órdenes $O' \subseteq O$ que maximicen el beneficio total respetando límites de ola de trabajo y disponibilidad de productos.
    
    \item \textbf{Variante 2: Selección de Pasillos con Órdenes Fijas}
    
    Dado un conjunto predefinido de órdenes fijas $O' \subseteq O$, seleccionar el conjunto mínimo de pasillos $A' \subseteq A$ que satisfaga toda la demanda de productos.
    
    \item \textbf{Variante 3: Selección de Órdenes con Generación de Columnas}
    
    Resolver el mismo problema que la Variante 1 utilizando técnicas avanzadas de generación de columnas para manejar problemas de gran escala.
\end{enumerate}

%====================================================================
\section{Desarrollo e implementación}
%====================================================================

Esta sección presenta el desarrollo matemático detallado de cada variante y su implementación computacional en Python.

\subsection{Variante 1: Selección de Órdenes con Pasillos Fijos}
\label{sec:variante1}

\begin{problem}
Dado un conjunto predefinido de pasillos $A_{fijo} \subseteq A$, seleccionar un subconjunto de órdenes $O' \subseteq O$ que:
\begin{itemize}
    \item Maximice el beneficio total
    \item Respete los límites de ola de trabajo $[L_w, U_w]$
    \item Pueda ser satisfecho por los productos disponibles en $A_{fijo}$
\end{itemize}
\end{problem}

\subsubsection{Formulación Matemática}

\textbf{Conjuntos e Índices:}
\begin{itemize}
    \item $O = \{0, 1, \ldots, o-1\}$: Conjunto de órdenes
    \item $I = \{0, 1, \ldots, i-1\}$: Conjunto de productos
    \item $A_{fijo} \subseteq A$: Conjunto de pasillos fijos predefinidos
\end{itemize}

\textbf{Parámetros:}
\begin{itemize}
    \item $c_{jt}$: Cantidad del producto $t$ en la orden $j$
    \item $d_{kt}$: Cantidad del producto $t$ disponible en el pasillo $k \in A_{fijo}$
    \item $\beta_j$: Beneficio de la orden $j$
    \item $L_w, U_w$: Límites de ola de trabajo
\end{itemize}

\textbf{Variables de Decisión:}
\begin{itemize}
    \item $x_j \in \{0,1\}$: Variable binaria que indica si la orden $j$ es seleccionada
\end{itemize}

\textbf{Función Objetivo:}
\begin{equation}
\max \sum_{j \in O} \beta_j \cdot x_j
\end{equation}

\textbf{Restricciones:}

\begin{align}
L_w &\leq \sum_{j \in O} \sum_{t \in I} c_{jt} \cdot x_j \leq U_w \label{eq:wave_limits_v1} \\
\sum_{j \in O} c_{jt} \cdot x_j &\leq \sum_{k \in A_{fijo}} d_{kt}, \quad \forall t \in I \label{eq:product_availability_v1} \\
x_j &\in \{0,1\}, \quad \forall j \in O \label{eq:binary_v1}
\end{align}

\subsubsection{Implementación en Python}

La implementación se encuentra en \texttt{desafioparte1.py} y utiliza PuLP para la optimización:

\begin{lstlisting}[language=Python, caption=Implementación clave de la Variante 1]
# Variables de decisión
x_vars = pulp.LpVariable.dicts("x", range(K_ordenes), cat=pulp.LpBinary)

# Función objetivo: maximizar beneficio de órdenes seleccionadas
modelo += pulp.lpSum(beneficio_orden_j[j] * x_vars[j] 
                     for j in range(K_ordenes))

# Restricción de límites de ola de trabajo
total_items_expr = pulp.lpSum(
    matriz_W[j][i] * x_vars[j] 
    for j in range(K_ordenes) 
    for i in range(I_items)
)
modelo += total_items_expr >= L_w
modelo += total_items_expr <= U_w

# Restricciones de disponibilidad por producto
for i in range(I_items):
    disponible_producto_i = sum(matriz_S[k][i] for k in A_fijo_indices)
    demanda_producto_i = pulp.lpSum(
        matriz_W[j][i] * x_vars[j] for j in range(K_ordenes)
    )
    modelo += demanda_producto_i <= disponible_producto_i
\end{lstlisting}

\subsection{Variante 2: Selección de Pasillos con Órdenes Fijas}
\label{sec:variante2}

\begin{problem}
Dado un conjunto predefinido de órdenes fijas $O' \subseteq O$, seleccionar el conjunto mínimo de pasillos $A' \subseteq A$ que pueda satisfacer toda la demanda de productos de las órdenes fijas.
\end{problem}

\subsubsection{Formulación Matemática}

\textbf{Conjuntos e Índices:}
\begin{itemize}
    \item $O' \subseteq O$: Conjunto de órdenes fijas predefinidas
    \item $I = \{0, 1, \ldots, i-1\}$: Conjunto de productos
    \item $A = \{0, 1, \ldots, a-1\}$: Conjunto de pasillos disponibles
\end{itemize}

\textbf{Parámetros:}
\begin{itemize}
    \item $c_{jt}$: Cantidad del producto $t$ en la orden $j \in O'$
    \item $d_{kt}$: Cantidad del producto $t$ disponible en el pasillo $k$
    \item $\alpha_k$: Costo de utilizar el pasillo $k$
    \item $D_t = \sum_{j \in O'} c_{jt}$: Demanda total del producto $t$
\end{itemize}

\textbf{Variables de Decisión:}
\begin{itemize}
    \item $y_k \in \{0,1\}$: Variable binaria que indica si el pasillo $k$ es seleccionado
\end{itemize}

\textbf{Función Objetivo:}
\begin{equation}
\min \sum_{k \in A} \alpha_k \cdot y_k
\end{equation}

\textbf{Restricciones:}

\begin{align}
\sum_{k \in A} d_{kt} \cdot y_k &\geq D_t, \quad \forall t \in I \label{eq:demand_satisfaction_v2} \\
y_k &\in \{0,1\}, \quad \forall k \in A \label{eq:binary_v2}
\end{align}

\subsubsection{Implementación en Python}

La implementación en \texttt{desafioparte2.py} incluye validaciones para verificar que las órdenes fijas cumplan con los límites de ola de trabajo:

\begin{lstlisting}[language=Python, caption=Implementación clave de la Variante 2]
# Variables de decisión
y_vars = pulp.LpVariable.dicts("y", range(J_pasillos), cat=pulp.LpBinary)

# Función objetivo: minimizar costo de pasillos seleccionados
modelo += pulp.lpSum(costo_pasillo_k[k] * y_vars[k] 
                     for k in range(J_pasillos))

# Verificación previa de límites de ola de trabajo
total_items_ordenes_fijas = sum(
    matriz_W[j][i] for j in ordenes_fijas_indices 
                   for i in range(I_items)
)
if not (L_w <= total_items_ordenes_fijas <= U_w):
    print("ADVERTENCIA: Órdenes fijas no cumplen límites de ola")

# Restricciones de satisfacción de demanda por producto
for i in range(I_items):
    demanda_total_producto_i = sum(
        matriz_W[j][i] for j in ordenes_fijas_indices_validos
    )
    oferta_producto_i = pulp.lpSum(
        matriz_S[k][i] * y_vars[k] for k in range(J_pasillos)
    )
    modelo += oferta_producto_i >= demanda_total_producto_i
\end{lstlisting}

\subsection{Variante 3: Selección de Órdenes con Generación de Columnas}
\label{sec:variante3}

Esta variante utiliza la técnica de \emph{Generación de Columnas} para resolver el mismo problema que la Variante 1, pero con mayor escalabilidad para instancias grandes.

\subsubsection{Enfoque de Generación de Columnas}

La idea central es reformular el problema introduciendo \emph{patrones} de órdenes que pueden ser servidos por cada pasillo.

\textbf{Definición de Patrones:}

Un patrón $p$ para un pasillo $a$ es un subconjunto de órdenes $O_p \subseteq O$ tal que:
\begin{itemize}
    \item El pasillo $a$ tiene productos suficientes para satisfacer todas las órdenes en $O_p$
    \item La cantidad total de productos en $O_p$ está dentro de los límites $[L_w, U_w]$
\end{itemize}

\subsubsection{Problema Maestro Restringido (RMP)}

\textbf{Variables de Decisión:}
\begin{itemize}
    \item $\lambda_p \in \{0,1\}$: Variable binaria que indica si el patrón $p$ es seleccionado
\end{itemize}

\textbf{Función Objetivo:}
\begin{equation}
\max \sum_{a \in A_{fijo}} \sum_{p \in P_a} v_p \cdot \lambda_p
\end{equation}

donde $v_p$ es el valor del patrón $p$ (suma de beneficios de las órdenes incluidas).

\textbf{Restricciones:}

\begin{align}
\sum_{a \in A_{fijo}} \sum_{p \in P_a} a_{jp} \cdot \lambda_p &\leq 1, \quad \forall j \in O \label{eq:order_once} \\
\sum_{p \in P_a} \lambda_p &\leq 1, \quad \forall a \in A_{fijo} \label{eq:one_pattern_per_aisle}
\end{align}

donde $a_{jp} = 1$ si la orden $j$ está incluida en el patrón $p$.

\subsubsection{Subproblemas de Pricing}

Para cada pasillo $a \in A_{fijo}$, se resuelve un subproblema que busca generar nuevos patrones rentables:

\textbf{Variables de Decisión:}
\begin{itemize}
    \item $z_j \in \{0,1\}$: Variable binaria que indica si la orden $j$ se incluye en el nuevo patrón
\end{itemize}

\textbf{Función Objetivo:}
\begin{equation}
\max \sum_{j \in O} (\beta_j - \pi_j) \cdot z_j
\end{equation}

donde $\pi_j$ son las variables duales del RMP.

\textbf{Restricciones:}

\begin{align}
\sum_{j \in O} c_{jt} \cdot z_j &\leq d_{at}, \quad \forall t \in I \label{eq:aisle_capacity} \\
L_w \leq \sum_{j \in O} \sum_{t \in I} c_{jt} \cdot z_j &\leq U_w \label{eq:wave_limits_subproblem}
\end{align}

\subsubsection{Implementación en Python}

El archivo \texttt{desafioparte3.py} implementa el algoritmo completo:

\begin{lstlisting}[language=Python, caption=Algoritmo principal de generación de columnas]
def algoritmo_generacion_columnas():
    # Inicializar RMP con patrones triviales
    patrones_iniciales = generar_patrones_iniciales()
    
    for iteracion in range(MAX_ITERACIONES_CG):
        # Resolver RMP relajado
        status_rmp, obj_rmp, duales = resolver_rmp_relajado(patrones_actuales)
        
        # Resolver subproblemas de pricing
        nuevos_patrones = []
        for pasillo_a in A_fijo:
            patron_nuevo = resolver_subproblema_pricing(pasillo_a, duales)
            if patron_nuevo and costo_reducido(patron_nuevo) > EPSILON:
                nuevos_patrones.append(patron_nuevo)
        
        # Verificar condición de parada
        if not nuevos_patrones:
            print("Optimalidad alcanzada")
            break
            
        # Agregar nuevos patrones al RMP
        patrones_actuales.extend(nuevos_patrones)
    
    # Resolver RMP final con variables enteras
    return resolver_rmp_entero(patrones_actuales)

def resolver_subproblema_pricing(pasillo_a, duales_pi_k):
    """Resuelve el subproblema de pricing para un pasillo específico."""
    modelo_sub = pulp.LpProblem(f"Subproblema_Pasillo_{pasillo_a}", 
                                pulp.LpMaximize)
    
    # Variables: órdenes a incluir en el patrón
    y_vars = pulp.LpVariable.dicts(f"y_p{pasillo_a}", 
                                   range(K_ordenes), cat=pulp.LpBinary)
    
    # Objetivo: maximizar costo reducido
    coeficientes_obj = []
    for k in range(K_ordenes):
        valor_orden_k = sum(matriz_W[k][i] for i in range(I_items))
        coeficientes_obj.append(valor_orden_k - duales_pi_k[k])
    
    modelo_sub += pulp.lpSum(coeficientes_obj[k] * y_vars[k] 
                             for k in range(K_ordenes))
    
    # Restricciones de disponibilidad por producto
    for i in range(I_items):
        modelo_sub += pulp.lpSum(matriz_W[k][i] * y_vars[k] 
                                 for k in range(K_ordenes)) <= stock_pasillo_a[i]
    
    # Restricciones de límites de ola de trabajo
    total_items = pulp.lpSum(matriz_W[k][i] * y_vars[k] 
                             for k in range(K_ordenes) 
                             for i in range(I_items))
    modelo_sub += total_items >= L_w
    modelo_sub += total_items <= U_w
    
    # Resolver subproblema
    modelo_sub.solve()
    
    if modelo_sub.status == pulp.LpStatusOptimal:
        patron_nuevo = [k for k in range(K_ordenes) if y_vars[k].value() > 0.5]
        return patron_nuevo
    return None
\end{lstlisting}

\subsection{Sistema de Carga de Datos}

Todas las implementaciones utilizan un sistema común desarrollado en \texttt{leer\_archivo.py} y \texttt{modelos.py}:

\begin{lstlisting}[language=Python, caption=Sistema de carga de datos]
def cargar_datos_instancia_desafio(nombre_archivo):
    """Carga datos desde archivo de instancia del formato especificado."""
    with open(nombre_archivo, 'r') as archivo:
        # Primera línea: K I J
        K, I, J = map(int, archivo.readline().strip().split())
        
        # Cargar órdenes (formato disperso)
        ordenes_datos = []
        for _ in range(K):
            linea = archivo.readline().strip().split()
            orden_id = int(linea[0])
            productos = {}
            for item in linea[2:]:  # Skip num_productos
                prod_id, cantidad = map(int, item.split(':'))
                productos[prod_id] = cantidad
            ordenes_datos.append((orden_id, productos))
        
        # Cargar pasillos (formato disperso)
        pasillos_datos = []
        for _ in range(J):
            linea = archivo.readline().strip().split()
            pasillo_id = int(linea[0])
            x, y = int(linea[1]), int(linea[2])
            productos = {}
            for item in linea[4:]:  # Skip num_productos
                prod_id, cantidad = map(int, item.split(':'))
                productos[prod_id] = cantidad
            pasillos_datos.append((pasillo_id, x, y, productos))
    
    return K, I, J, ordenes_datos, pasillos_datos

def transformar_datos_dispersos_a_densos(ordenes_datos, pasillos_datos, I_items):
    """Convierte datos dispersos a matrices densas para optimización."""
    K_ordenes = len(ordenes_datos)
    J_pasillos = len(pasillos_datos)
    
    # Matriz de órdenes: W[j][i] = cantidad del producto i en orden j
    matriz_W = [[0 for _ in range(I_items)] for _ in range(K_ordenes)]
    for j, (_, productos) in enumerate(ordenes_datos):
        for prod_id, cantidad in productos.items():
            if 0 <= prod_id < I_items:
                matriz_W[j][prod_id] = cantidad
    
    # Matriz de pasillos: S[k][i] = cantidad del producto i en pasillo k
    matriz_S = [[0 for _ in range(I_items)] for _ in range(J_pasillos)]
    for k, (_, _, _, productos) in enumerate(pasillos_datos):
        for prod_id, cantidad in productos.items():
            if 0 <= prod_id < I_items:
                matriz_S[k][prod_id] = cantidad
    
    return matriz_W, matriz_S
\end{lstlisting}

\subsection{Herramientas y Tecnologías Utilizadas}

\begin{itemize}
    \item \textbf{Python 3.x}: Lenguaje de programación principal
    \item \textbf{PuLP}: Librería de programación lineal que proporciona una interfaz de alto nivel para solvers de optimización
    \item \textbf{Solver CBC}: Solver por defecto (Coin-or Branch and Cut) para problemas de programación lineal entera mixta
    \item \textbf{Diseño modular}: Separación clara entre carga de datos, modelado y resolución
\end{itemize}

%====================================================================
\section{Dificultades y aprendizajes}
%====================================================================

Durante el desarrollo de este trabajo se enfrentaron diversas dificultades que proporcionaron valiosos aprendizajes sobre optimización combinatoria y su implementación práctica.

\begin{enumerate}
    \item \textbf{Comprensión del contexto del problema}
    
    La transición conceptual desde la descripción abstracta del problema hacia su interpretación en el contexto real de optimización de almacenes requirió un análisis cuidadoso de la semántica de variables y restricciones. La terminología inicial genérica dificultaba la comprensión del problema subyacente.
    
    \emph{Aprendizaje:} La importancia de contextualizar problemas de optimización en dominios específicos para una mejor comprensión y comunicación de resultados.

    \item \textbf{Manejo del formato de datos dispersos}
    
    El formato de entrada, donde solo se especifican productos con cantidad mayor a cero, requirió desarrollar rutinas de conversión a representaciones densas. Esta transformación es crítica para la eficiencia computacional.
    
    \emph{Aprendizaje:} La representación de datos impacta significativamente la complejidad de implementación y el rendimiento de los algoritmos.

    \item \textbf{Validación de factibilidad en la Variante 2}
    
    La implementación reveló la necesidad de validar que las órdenes fijas cumplan con los límites de ola de trabajo antes de proceder con la optimización. Sin esta validación, el modelo podría ser infactible desde el inicio.
    
    \emph{Aprendizaje:} La importancia de incluir verificaciones de factibilidad y validaciones de entrada en implementaciones prácticas.

    \item \textbf{Complejidad de la Generación de Columnas}
    
    La implementación de la Variante 3 requirió comprensión profunda de la descomposición de Dantzig-Wolfe y la coordinación entre el problema maestro y los subproblemas de pricing. La gestión de información dual y la generación de patrones iniciales presentaron desafíos considerables.
    
    \emph{Aprendizaje:} Las técnicas avanzadas de optimización requieren implementación cuidadosa pero ofrecen ventajas significativas para problemas de gran escala.

    \item \textbf{Interpretación y validación de resultados}
    
    Más allá de obtener soluciones óptimas, fue necesario desarrollar métodos para interpretar y validar resultados en el contexto del problema original, verificando que las soluciones respeten todas las restricciones y tengan sentido operacional.
    
    \emph{Aprendizaje:} La validación post-optimización es tan importante como la formulación correcta del problema.
\end{enumerate}

\textbf{Principales aprendizajes obtenidos:}

\begin{itemize}
    \item \textbf{Diseño modular}: La separación clara entre carga de datos, definición del modelo, resolución y análisis facilitó significativamente el desarrollo y depuración.
    
    \item \textbf{Experimentación incremental}: Comenzar con instancias pequeñas y controladas permitió verificar la corrección antes de abordar casos complejos.
    
    \item \textbf{Flexibilidad de PuLP}: La librería demostró ser efectiva para prototipado rápido con una curva de aprendizaje razonable.
    
    \item \textbf{Técnicas avanzadas}: La generación de columnas ilustró cómo las técnicas de descomposición hacen tratables problemas que serían prohibitivamente grandes con enfoques directos.
    
    \item \textbf{Análisis de sensibilidad}: Los experimentos con diferentes configuraciones proporcionaron insights valiosos sobre el comportamiento de las soluciones óptimas.
\end{itemize}

\end{document}
